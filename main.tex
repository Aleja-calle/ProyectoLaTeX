\documentclass{article}
\usepackage[utf8]{inputenc}
\usepackage[spanish]{babel}
\usepackage{listings}
\usepackage{graphicx}
\graphicspath{ {images/} }
\usepackage{cite}

\begin{document}

\begin{titlepage}
    \begin{center}
        \vspace*{1cm}
            
        \Huge
        \textbf{Informe}
            
        \vspace{0.5cm}
        \LARGE
        Desafío pasar un objeto de un punto A a un punto B
            
        \vspace{1.5cm}
            
        \textbf{Alejandra Calle Vasquez}
            
        \vfill
            
        \vspace{0.8cm}
            
        \Large
        Despartamento de Ingeniería Electrónica y Telecomunicaciones\\
        Universidad de Antioquia\\
        Medellín\\
        Marzo de 2021
            
    \end{center}
\end{titlepage}


\newpage
\section{Descripción del desafío.}\label{descripción}
Este desafío se trata de posicionar dos tarjetas que están en una posición incial y llevarlas a su posición final solo siguiendo las instrucciones que se dan. 
\section{posición inicial y final.}\label{Posiciones}
La posición inicial de las cartas empiezan estando debajo de la hoja y la posicón final es con las cartas equilibradas formando una casita encima de la hoja.
\section{Pasos para realizar el desafío} \label{Pasos}
\textbf{Primer paso: }
observar la posición inicial de la actividad
\vspace{0.3cm}

\textbf{Segundo paso: }
levantar la hoja y ubicarla a un lado de modo que puedas ver las tarjetas que hay debajo de ella.
\vspace{0.3cm}

\textbf{Cuarto paso: }
con una sola mano (la que más confianza tengas) tomar las dos tarjetas.
\vspace{0.3cm}

\textbf{Quinto paso: }
apoyar las tarjetas de manera vertical(como creando una columna) paradas encima de la hoja sin soltarlas.
\vspace{0.3cm}

\textbf{Sexto paso: } 
asegurarse de que estén correctamente alineadas entre ellas.
\vspace{0.3cm}

\textbf{Séptimo paso: }
ubicar el dedo pulgar en el lateral izquierdo de las tarjetas.
\vspace{0.3cm}

\textbf{Octavo paso: }
ubicar el dedo índice en la parte superior de las tarjetas y los dedos corazón anular y meñique en el lateral derecho de las tarjetas.
\vspace{0.3cm}

\textbf{Noveno paso: }
arrastrar poco a poco la parte inferior de una de las tarjetas hacia ti sin despegar la parte de arriba de la otra tarjeta tratando de formar un triangulo
\vspace{0.3cm}

\textbf{Décimo paso: }
tratar de equilibrar las tarjetas de manera que al soltarlas queden equilibradas formando un triángulo equilátero o un ángulo de 40° aproximadamente 
\vspace{0.3cm}

\newpage
\section{Conclusiones}\label{Conclusiones}
Para realizar esta actividad de forma correcta debemos tener muy en cuenta las ambigüedades que puedan existir en nuestro idioma y que debemos ser lo más claros posibles para poder dar a entender que es lo que queremos y que sea sencillo llegar a ello.
\end{document}
